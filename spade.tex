\documentclass{article}
\usepackage{geometry}
\usepackage{natbib}
\bibliographystyle{apalike}
\usepackage{amsmath,amssymb}
\usepackage{mathtools}
\usepackage{multirow}
\usepackage{float,endfloat}
\usepackage{appendix}
\usepackage{graphicx}
\usepackage{algorithm}
\usepackage[noend]{algpseudocode}

\DeclareMathOperator*{\argmin}{argmin}

\title{Stock assessment using the\\ McKendrick-Gurtin-MacCamy-Murphy\\ partial differential equation}

%Plasticity and stock assessment:\\
%the partial differential approach}
%Mckendrick style fishery models and application to density-dependent growth ...
%Density-dependent growth effects on productivity and capacity: the case of Australian barramundi \emph{Lates calcarifer}}
\author{
  Alexander Campbell\\
  Queensland Department of Agriculture and Fisheries\\
  Brisbane, Australia
  \and
  Oscar Angulo\\
  Departamento de Matem\'{a}tica Aplicada\\
  Universidad de Valladolid\\
  Valladolid, Spain
  \and
  Tomislav Buric\\
  Faculty of Electrical Engineering and Computing,\\
  University of Zagreb, Croatia\\
  School of Mathematics and Physics\\
  University of Queensland\\
  Brisbane, Australia
}

\floatstyle{boxed}
\newfloat{code}{thp}{lop}
\floatname{code}{Code}
%\DeclareDelayedFloat{code}{Codes}

\floatstyle{boxed}
\newfloat{derivation}{thp}{lop}
\floatname{derivation}{Derivation}
%\DeclareDelayedFloat{derivation}{Derivations}

\begin{document}
\maketitle

\begin{abstract}
In 1926 Anderson Gray McKendrick introduced an equation that describes the death process, in continuous-time, of an age-structured population.  In this article we trace the way this fundamental equation has been extended and generalised in the intervening years into a form which encapsulates much of what we call modern stock assessment.  In a sense it is the prototypical stock assessment model - at least for species with an extended breeding period - of which all matrix-based approaches are an approximation.  We introduce numerical and analytical solutions to the equation that turn this elegant mathematical framework into a practical modelling tool, and demonstrate this using an Australian barramundi (\emph{Lates calcarifer}) fishery.  

%We explore the insights this framework provides on many of the problems facing modern stock assessment, including growth plasticity.
\end{abstract}

%, that is to say, into a tool which can be used to inform fishery management.
%. We discuss how this framework is best suited to dealing with many of the challenges that still face stock assessment, such as plasticity, 


%Given that the final mathematical form of this model was attained more than thirty years ago, we speculate that two key technical issues have been preventing its 

%the reason for its absence from the fishery modelling literature is because of a requirement for numerical solutions. 

%Given that this kind of model has been well known in ecology for the last thirty years, we speculate that the reason for its absence from the fishery modelling literature is because of the necessetya
%\end{abstract}
  

\section{Introduction}

what would a fishery model look like in this formalism?

Providing there is sufficient information available to calibrate them, population models that explicitly represent the size of the individual are significantly more realistic than those that lump this dimension into one or a few state variables. This enhanced realism is particularly important for populations which experience a forcing that acts along the dimension, for example a time-dependent mortality process that varies as a function of size. `Size' is being used as a shorthand for any physiologically important dimension, for example age, length, weight, body mass etc. These models are commonly referred to as physiologically structured population models \citep{Metz2014}. 

% - but .. profound (cite van Sickle) 
Here we present a size structured population model wherein the population is subject to an exogenous mortality process, namely fishing, and for which parameters are estimated by minimising the discrepancy between predictions of the model and empirical data. This is also the goal of the applied science of `stock assessment' \citep{Hilborn1992}, where the estimated parameters are used to guide the management of a fishery. The methods we present here, however, differ somewhat from those used in stock assessment, and bear more resemblance to the mathematical ecology literature. In particular, we represent the physiological dimension mathematically as a continuum, rather than as a discrete set of states as is done almost exclusively in stock assessment. The decision to represent both time and size as a continuum means that we are dealing with partial differential equations, and brings with it some challenges as well as some advantages. The aim of this paper is to show one of way of tackling the challenges, and to highlight some of the advantages. While some things are ostensibly harder in this paradigm, such as the need to solve differential equations using numerical techniques, some things are possible that are not in the discrete paradigm, and some things actually become easier. Perhaps the main advantage though is conceptual clarity. 

The approach is demonstrated by fitting the model to data from a barramundi fishery in Queensland, Australia.

\clearpage
\section{From Malthus to Murphy}\label{sec:model}
The simplest population model follows the Malthusian principle 
\begin{equation}
\dot{N} = \alpha N
\end{equation}
which implies exponential growth and is clearly inappropriate for real world populations that must compete for resources.  \citet{Verhulst1845,Verhulst1847} understood that the growth rate should depend on total population size and achieved this in the simplest possible way,
\begin{equation}
\dot{N} = (\alpha_0-\beta_0 N)N
\end{equation}
so that the rate of increase decreases as the population grows, and can be negative for large $N$. This model can be solved exactly,
\begin{equation}
N(t) = \frac{ \zeta}{1 + \left( \frac{\zeta}{N(0)}-1\right)e^{-\alpha_0 t}}
\end{equation}
describing the evolution of a population towards the stable equilibrium point $\zeta=\alpha_0/\beta_0$.

A limitation of the Verhulstian model is that it does not consider the age of the population.  The first explicit reference to age in a differential equation was made by \citet{McKendrick1926}\footnote{\citet{Lotka1925} clearly understood the concept and many of the consequences but did not write down the equation from which they flow. In ecology it is often referred to as the \citet{vonFoerster1959} equation.}, 
\begin{equation}
\frac{\partial n(a,t)}{\partial t} + \frac{\partial n(a,t)}{\partial a} = -\mu(a,t) n(a,t)
\end{equation}
where $n(a,t)$ is the number of individuals of age $a$ at time $t$, and $\mu$ is rate at which they die per unit of population. 

This equation requires a boundary condition on $n(a,t)$ in both $t$ and $a$. The initial condition,
\begin{equation}
n(a,0) = f(a)
\end{equation}
says that the population has a given initial age distribution. The other boundary condition is the renewal equation, aka the birth rate
\begin{equation}
n(0,t) = \int_a b(a)n(a,t)\,da
\end{equation}
where $b(a)$ is the birth rate at age $a$.  Note that this is an unusual boundary condition for a PDE in that it is prescribed but must be computed from the population at time t.

However the McKendrick model has the same problem as Malthus - the death process is not dependent on the population.  Clearly a Verhulstian effect is needed.  \citet{Sinko1967} give a rigorous derivation of the McKendrick model.. and they explain its relationship to the Verhulst model by integrating equation \ref{eq:1} over age and then setting $\mu$ to .. but they do not actually write down the PDE with population dependence inside it, nor do they seem to realise such a model is a different class from \ref{eq:1} and that the existence and uniqueness proof given in \citet{Sinko1968}.  This was done by \citet{Gurtin1974},
\begin{equation}
\frac{\partial n(a,t)}{\partial t} + \frac{\partial n(a,t)}{\partial a} = -\mu(a,N,t) n(a,t)
\end{equation}



Very sensible to have an age dependent death rate.

Population dynamics are based on the McKendrick partial differential equation \citep{McKendrick1926}, generalised to accommodate density dependence \citep{Gurtin1974} and size structured dynamics \citep{Murphy1983}, and with a time-dependent mortality function:
\begin{subequations}
\label{eq:1}
\begin{align}
\frac{\partial u(x,t)}{\partial t} + \frac{\partial [g(x)u(x,t)]}{\partial x} &=
-z(x,U(t),t)u(x,t) \label{eq:1.1}\\
g(0)u(0,t) &=  \int_0^{\omega} b(x) u(x,t)\,dx\label{eq:1.2}\\ 
U(t) &= \int_0^{\omega} u(x,t)\, dx
\end{align}
\end{subequations}
where $x$ is the size of the individual, $\omega$ is the maximum size, and $b$, $g$ and $z$ represent the processes of birth, growth and death respectively. The birth and growth processes are parameterised as      
\begin{subequations}
  \begin{align}
    g(x) &= \kappa (\omega - x)\label{seq:2b}\\
    b(x) &= \left(\alpha_1 x + \alpha_2 x^2\right)\label{seq:2c},%\exp\left(\psi \cos(t-\phi)\right)/ 2\pi I_0(\psi)
  \end{align}
\end{subequations}
while the mortality process is given by 
\begin{equation}
  z(x,U(t),t) = \beta + \gamma U(t) + s(x)f(t)\label{seq:2a},
\end{equation}
where $\beta$ and $\gamma$ are respectively density-independent, and density-dependent, natural mortalities, $s(x)$ is the `selectivity' function (different sized fish are differentially selected by the fishing gear), given by
\begin{equation}
  s(x) = \exp(-(x-s_1)^2 / s_2)
\end{equation}
and $f(t)$ is fishing mortality, 
\begin{equation}
  f(t) = \iota e(t)  
\end{equation}
where $\iota$ is the catchability and $e(t)$ is the observed effort. The fishing process produces a catch,
\begin{equation}
  c(x,t) = w(x)s(x)f(t)u(x,t)
\end{equation}
where $w(x)$ is the weight of a fish of size $x$, given by
\begin{equation}
  w(x)=w_1 x^{w_2}.
\end{equation}
This completes the specification of a simple fishery system. The biological component of this system is a slight variant of that analysed by \citet{Murphy1983} - simpler in that growth and births are not density dependent, but more complex in that growth is size-dependent (von Bertalanffy). Another way to think of it is as the simplest modification of the first model in \citet{Gurtin1978} needed to handle size-dependent growth.\\

\clearpage
\section{Numerical solution}
Equation \ref{eq:1} is a nonlinear hyperbolic partial differential equation with a nonlinear and non-local boundary condition, and in general must be solved numerically. As with all Mckendrick style PDEs it can be reduced to a coupled ODE problem by integrating along the characteristic curves \citep{Kot2001}. Following \citet{Angulo2004}, define
\begin{equation}
  z^*(x,U,t)=z(x,U,t)+g_x(x)
\end{equation}
so that \ref{eq:1.1} has the form
\begin{equation}\label{eq:2}
  u_t(x,t) + g(x)u_x(x,t) = -z^*(x,U,t)u(x,t).
\end{equation}
Now denote by $x(t;t^*,x^*)$ the characteristic curve of Equation \ref{eq:2} that takes the value $x^*$ at time $t^*$, which is the solution to the following initial value problem
\begin{equation}\label{eq:2.2}
  \begin{cases}
    \frac{d}{dt} x(t;t^*,x^*)=g(x(t;t^*,x^*)), & t\geq t^*\\
    x(t^*;t^*,x^*)=x^* &. 
  \end{cases}
\end{equation}
Next, define the function
\begin{equation}
  r(t;t^*,x^*)=u(x(t;t^*,x^*),t)
\end{equation}
which satisfies the folloiwng initial value problem
\begin{equation}
  \begin{cases}
    \frac{d}{dt} r(t;t^*,x^*)=-z^*(x(t;t^*,x^*),U,t)r(t;t^*,x^*), & t\geq t^*,\\
    r(t^*;t^*,x^*) = u(x^*,t^*), &
  \end{cases}
\end{equation}
and therefore can be represented by 
\begin{equation}\label{eq:2.5}
  r(t;t^*,x^*)=u(x^*,t^*)\exp\left(-\int_{t^*}^{t}z^*(x(\tau;t^*,x^*),U,t)\,d\tau\right).
\end{equation}
The coupled problems \ref{eq:2.2} and \ref{eq:2.5} are then simultaneously integrated together with boundary condition \ref{eq:1.1}. We use a slightly modified version of the second order numerical scheme of \citet{Angulo2014}, given in Algorithm \ref{alg:base}.
\begin{algorithm}
  \caption{Model PDE numerical solution}\label{alg:base}
  \begin{algorithmic}
    \State $X_0^{n+1} \gets 0$
    \State $X_{j+1}^{n+1} \gets X_j^n + k g(X_{j+1}^{n+1/2}), 0 \leq j \leq J$
    \State $U_{j+1}^{n+1} \gets U_j^n \exp\left(-k z^*(X_{j+1}^{n+1/2},\mathcal{Q}(\mathbf{X}^{n+1/2},\mathbf{U}^{n+1/2}),t^{n+1/2})\right), 0 \leq j \leq J$
    \State $U_0^{n+1} \gets \mathcal{Q}(\mathbf{X}^{n+1},b(\mathbf{U}^{n+1})) / g(X_{0}^{n+1})$
    \State $X_0^{n+1/2} \gets 0$
    \State $X_{j+1}^{n+1/2} \gets X_j^n + (k/2) g(X_{j+1}^{n+1/4}), 0 \leq j \leq J$
    \State $U_{j+1}^{n+1/2} \gets U_j^n \exp\left(-(k/2) z^*(X_{j+1}^{n+1/4},\mathcal{Q}(\mathbf{X}^{n+1/4},\mathbf{U}^{n+1/4}),t^{n+1/4})\right), 0 \leq j \leq J$
    \State $U_0^{n+1/2} \gets \mathcal{Q}(\mathbf{X}^{n+1/2},b(\mathbf{U}^{n+1/2})) / g(X_{0}^{n+1/2})$
    \State $X_{j+1}^{n+1/4} \gets X_j^n + (k/4) g(X_{j}^{n}), 0 \leq j \leq J$
    \State $U_{j+1}^{n+1/4} \gets U_j^n \exp\left(-(k/4) z^*(X_{j}^{n},\mathcal{Q}(\mathbf{X}^{n},\mathbf{U}^{n}),t^{n})\right), 0 \leq j \leq J$
    \State $U_0^{n+1/4} \gets \mathcal{Q}(\mathbf{X}^{n+1/4},b(\mathbf{U}^{n+1/4})) / g(X_{0}^{n+1/4})$
  \end{algorithmic}
\end{algorithm}
Following \citet{Angulo2014} we use an adaptive mesh that keeps the number of nodes along the size dimension constant as the algorithm steps forward: at time $t^{n+1}$ we remove the grid node $X_l^{n+1}$ that satisfies
\begin{equation}
  |X_{l+1}^{n+1} - X_{l-1}^{n+1}| = \text{min}_{1\leq j\leq J+1} |X_{j+1}^{n+1} - X_{j-1}^{n+1}|
\end{equation}
See \citet{Angulo2014} for convergence analysis and numerical experiments. 

%The R code for the scheme, implemented for the particular model outlined in section \ref{sec:model}, is given in Appendix \ref{app:ns}.

\clearpage
\section{Parameter Estimation}
The model is calibrated against fishery catch-at-size data. This comes from two sources: commercial catch data (a scalar time series representing removals from the population), and scientific monitoring program data on the size structure of the commercial catch. These are combined to produce the `observation', $c(x,t)$: the catch rate (in kilograms per centimetre) at time $t$ and size $x$. It is this observation to which the model is fit:
\begin{equation}
H(\theta) = \int_0^T \int_0^\omega |s(x)w(x)f(t)u(x,t) - c(x,t)| \,dx\,dt
\end{equation}
Optimisation is a hard problem in general, and optimisation routines that make use of the gradient of the objective function (with respect to the model parameters) are usually significantly more efficient. The gradient of $H$ with respect to $\theta$ is given by
\begin{equation}
%\frac{d H(\theta)}{d\theta}=
\int_0^T \int_0^\omega
\left(s(x)w(x)f_\theta(t)u(x,t)+s(x)w(x)f(t)p(x,t)\right)\times 
\begin{cases}
-1, &  s(x)w(x)f(t)u(x,t) < c(x,t)\\
\phantom{-}1, & s(x)w(x)f(t)u(x,t)>c(x,t) 
\end{cases}\,dx\,dt
\end{equation}
where $p(x,t)=u_\theta(x,t)$ are solutions of the the sensitivity-PDE \citep{Borggaard1997,Li2004}: %,Stanley2002}.
\begin{equation}
  \frac{d F}{d\theta} = F_\theta + F_u p + F_{u_t} p_t + F_{u_x} p_x = 0
\end{equation}
with
\begin{equation}
  F=u_t + \left[g(x;\theta)u\right]_x + z(x,U,t;\theta)u = 0,
\end{equation}
so that the sensitivity PDE is given by
\begin{equation}
  p_t + g_\theta u_x + g p_x + g_x p + {g_x}_\theta u + z p + \left(z_\theta + z_U P\right)u = 0
\end{equation}
where
\begin{equation}
  P=U_\theta= \int_0^\omega p(x,t;\theta)\,dx,
\end{equation}
with boundary condition
\begin{equation}
  g(0;\theta)_\theta u(0,t;\theta) + g(0;\theta) p(0,t;\theta) = \int_0^\omega b_\theta(x;\theta)u(x,t;\theta) + b(x;\theta)p(x,t;\theta) \,dx .
\end{equation}

Substituting in the birth, growth and death functions, we get the following sensitivity PDEs
\begin{subequations}
  \begin{align}
  \frac{d F}{d \alpha_1} &= p_t + \kappa(\omega-x)p_x - \kappa p + \left(\beta+\gamma U+s(x)\iota e(t)\right)p + \gamma P u = 0\\
  \frac{d F}{d \alpha_2} &= p_t + \kappa(\omega-x)p_x - \kappa p + \left(\beta+\gamma U+s(x)\iota e(t)\right)p + \gamma P u = 0\\
  \frac{d F}{d \beta} &= p_t + \kappa(\omega-x)p_x - \kappa p + \left(\beta+\gamma U+s(x)\iota e(t)\right)p + \left(1+ \gamma P\right)u = 0 \\
  \frac{d F}{d \gamma} &= p_t + \kappa(\omega-x)p_x - \kappa p + \left(\beta+\gamma U+s(x)\iota e(t)\right)p + \left(U+ \gamma P\right)u = 0 \\
  \frac{d F}{d \kappa} &= p_t + (\omega-x)u_x + \kappa(\omega-x)p_x - \kappa p - u + \left(\beta+\gamma U+s(x)\iota e(t)\right)p + \gamma P u = 0 \\
  \frac{d F}{d \omega} &= p_t + \kappa u_x + \kappa(\omega-x)p_x - \kappa p + \left(\beta+\gamma U+s(x)\iota e(t)\right)p + \gamma P u = 0 \\  
  \frac{d F}{d \iota} &= p_t + \kappa(\omega-x)p_x - \kappa p + \left(\beta+\gamma U+s(x)\iota e(t)\right)p + \left(s(x)e(t)+\gamma P\right) u = 0 
  \end{align}
\end{subequations}
The boundary conditions for each case are given by
\begin{subequations}
  \begin{align}
    \kappa\omega p(0,t) &= \int_0^\omega x u(x,t) + \alpha_1 x p(x,t) + \alpha_2 x^2 p(x,t) \,dx \\
    \kappa\omega p(0,t) &= \int_0^\omega x^2 u(x,t) + \alpha_1 x p(x,t) + \alpha_2 x^2 p(x,t) \,dx \\
    \kappa\omega p(0,t) &= \int_0^\omega \alpha_1 x p(x,t) + \alpha_2 x^2 p(x,t) \,dx \\
    \kappa\omega p(0,t) &= \int_0^\omega \alpha_1 x p(x,t) + \alpha_2 x^2 p(x,t) \,dx \\
    \omega u(0,t) + \kappa\omega p(0,t) &= \int_0^\omega \alpha_1 x p(x,t) + \alpha_2 x^2 p(x,t) \,dx \\
    \kappa u(0,t) + \kappa\omega p(0,t) &= \int_0^\omega \alpha_1 x p(x,t) + \alpha_2 x^2 p(x,t) \,dx \\
    \kappa\omega p(0,t) &= \int_0^\omega \alpha_1 x p(x,t) + \alpha_2 x^2 p(x,t) \,dx  
  \end{align}
\end{subequations}
These PDEs can be reduced to coupled ODE problems along characteristics in a manner analogous to the original PDE case. For all derivatives the size ODE (Equation \ref{eq:2.2}) remains unchanged. For $\iota$ the `population' ODE is
\begin{equation}
  \begin{split}
    q(t&;t^*,x^*)=p(x^*,t^*)\exp\left(-\int_{t^*}^t z^*(x(\tau;x^*,t^*),U(\tau),\tau)\,d\tau\right) -\\
    & \phantom{(;t^*,x^*)=p(x^*,t^*)}\exp\left(-\int_{t^*}^t z^*(x(\tau;x^*,t^*),U(\tau),\tau)\,d\tau\right)\times\\
    & \int_{t^*}^t m(x(\tau;t^*,x^*),P(\tau),u(x(\tau;t^*,x^*),\tau),\tau) \exp\left(\int_{t^*}^\tau z^*(x(\zeta;x^*,t^*),U(\zeta),\zeta)\,d\zeta\right)\,d\tau
  \end{split}
\end{equation}
where
\begin{equation}
  q(t;t^*,x^*)=p(x(t;t^*,x^*),t)
\end{equation}
and
\begin{equation}
  m(x,P(t),u(x,t),t)= \left(s(x)e(t)+\gamma P(t)\right) u(x,t).
\end{equation}

This is implemented with Algorithm \ref{alg:sens}. 
\begin{algorithm}
  \caption{Sensitivity PDE numerical solution for $\iota$}\label{alg:sens}
  \begin{algorithmic}
    \State $P_{j+1}^{n+1/2} \gets P_j^n \exp\left(-(k/2)z^*(X_{j}^{n},\mathcal{Q}(\mathbf{X}^{n},\mathbf{U}^{n}),t^{n})\right)-\exp\left(-(k/2)z^*(X_{j}^{n},\mathcal{Q}(\mathbf{X}^{n},\mathbf{U}^{n}),t^{n})\right)\times$\\ \hspace{1cm} $(k/2)\left(s(X_j^n)e(t^n)+\gamma\mathcal{Q}(\mathbf{X}^n,\mathbf{P}^n)\right)U_j^n$
    \State $P_0^{n+1/2} \gets \mathcal{Q}(\mathbf{X}^{n+1/2},b(\mathbf{P}^{n+1/2})) / g(X_0^{n+1/2})$
    \State $P_{j+1}^{n+1} \gets P_j^n \exp\left(-k z^*(X_{j+1}^{n+1/2},\mathcal{Q}(\mathbf{X}^{n+1/2},\mathbf{U}^{n+1/2}),t^{n+1/2})\right)-$\\ \hspace{1cm} $\exp\left(-k z^*(X_{j+1}^{n+1/2},\mathcal{Q}(\mathbf{X}^{n+1/2},\mathbf{U}^{n+1/2}),t^{n+1/2})\right)\times$\\ \hspace{1cm} $k\left(s(X_{j+1}^{n+1/2})e(t^{n+1/2})+\gamma\mathcal{Q}(\mathbf{X}^{n+1/2},\mathbf{P}^{n+1/2})\right)U_{j+1}^{n+1/2}\times$\\ \hspace{1cm} $ \exp\left((k/2)z^*(X_{j+1}^{n+1/4},\mathcal{Q}(\mathbf{X}^{n+1/4},\mathbf{U}^{n+1/4}),t^{n+1/4})\right)$
    \State $P_0^{n+1} \gets \mathcal{Q}(\mathbf{X}^{n+1},b(\mathbf{P}^{n+1})) / g(X_0^{n+1})$
  \end{algorithmic}
\end{algorithm}
Sensitivity PDE solutions for the other parameters follow analagously.

\subsection{Initial state}
If we assume that we have a complete fishing history, and that the initial state should therefore be unfished equilibrium, then it is possible to derive the starting condition analytically. 

The general technique is to introduce integral weighting functions to reduce the PDE to coupled ODEs. This was pioneered by \citet{Gurtin1978}, and has since been applied to more complicated models \citep{Murphy1983,Swart1994}. For our model, we start by defining
\begin{equation}
    V(t) = \int_0^{\omega} x u(x,t)\, dx
\end{equation}
and
\begin{equation}
    W(t) = \int_0^{\omega} x^2 u(x,t)\, dx .
\end{equation}
The model PDE (Equation \ref{eq:1}) is then integrated along the size dimension three times, the second and third time having been first multiplied by $x$ and $x^2$ respectively, producing three ODEs.\footnote{This reduction depends on a technical assumption that the initial data $u(x,0)$ has compact support so that $u(x,t)=0$ for sufficiently large $x$. This is obviously biologically realistic.} Details are contained in Appendix \ref{app:eq}. Eventually we obtain the system
\begin{subequations}
  \begin{align*}
    \frac{dU}{dt} &= -(\beta+\gamma U) U + \alpha_1 V + \alpha_2 W\\%\frac{e^{\psi \cos(\theta-\phi)}}{2\pi I_0(\psi)}\\
    \frac{d V}{d t} &= -(\beta+\gamma U)V + \kappa \omega U - \kappa V\\
    \frac{d W}{d t} &= -(\beta+\gamma U)W + 2\kappa\omega V - 2\kappa W\\
    %\frac{d \theta}{d t} &= 1
  \end{align*}
\end{subequations}
so that at equilibrium we have
\begin{subequations}
  \begin{align}
  (\beta+\gamma \bar{U}) \bar{U} &= \alpha_1 \bar{V} + \alpha_2 \bar{W}\label{seq:3a}\\
  (\beta + \gamma \bar{U})\bar{V}  &=  \kappa \omega \bar{U} - \kappa \bar{V}\label{seq:3b}\\
  (\beta+\gamma \bar{U})\bar{W}  &=  2\kappa\omega \bar{V} - 2\kappa \bar{W}\label{seq:3c}
  \end{align}
\end{subequations}

We can simplify \ref{seq:3b} to
\begin{equation}
  \bar{V} = \frac{\kappa \omega \bar{U} }{\beta + \gamma \bar{U} + \kappa}\label{eq:6}
\end{equation}
and \ref{seq:3c} to 
\begin{equation}
\bar{W} = \frac{2\kappa\omega\bar{V}}{\beta + \gamma\bar{U} + 2\kappa}\label{eq:7}
\end{equation}

Then let $Z=\beta + \gamma\bar{U}+\kappa$ so that
\begin{equation}\label{eqn:Z}
  \begin{split}
   (Z-\kappa) \frac{Z-\beta-\kappa}{\gamma} &= \frac{\alpha_1 \kappa\omega (Z-\beta-\kappa)}{\gamma Z} + \frac{\alpha_2 2 \kappa\omega \frac{\kappa\omega(Z-\beta-\kappa)}{\gamma Z}}{Z+\kappa}\\
%    \frac{Z-\kappa}{\gamma} &= \frac{\alpha_1 \kappa\omega}{\gamma Z} + \frac{\alpha_2 2 \kappa\omega \frac{\kappa\omega}{\gamma Z}}{Z+\kappa}\\
%    Z^2-\kappa Z - \alpha_1 \kappa\omega &= \frac{\alpha_2 2 \kappa\omega \kappa\omega}{Z+\kappa}\\
%    \left(Z+\kappa\right)\left(Z^2-\kappa Z - \alpha_1 \kappa\omega\right) &= \alpha_2 2 \kappa\omega \kappa\omega\\
%    Z^3 - \kappa Z^2 - \alpha_1\kappa\omega Z + \kappa Z^2 - \kappa^2 Z - \alpha_1 \kappa^2 \omega &= \alpha_2 2 \kappa\omega \kappa\omega\\
%    Z^3 - \alpha_1\kappa\omega Z - \kappa^2 Z - \alpha_1 \kappa^2 \omega &= \alpha_2 2 \kappa\omega \kappa\omega\\
    Z^3 - \kappa(\alpha_1\omega+\kappa)Z - \kappa\omega (\alpha_1\kappa + 2 \alpha_2\kappa\omega) &= 0
  \end{split}
\end{equation}

The real solution of this is
\begin{equation}
  Z = \frac{ \sqrt[3]{9 \alpha_1 \kappa^2 \omega+18 \alpha_2 \kappa^2 \omega^2+\kappa\zeta}}{3 \sqrt[3]{\frac23}} +  
  \frac{\sqrt[3]{\frac23}\,\kappa (\alpha_1  \omega+\kappa)}{\sqrt[3]{9 \alpha_1 \kappa^2 \omega+18 \alpha_2 \kappa^2 \omega^2+\kappa\zeta}}
\end{equation}  
where
\begin{equation}
\zeta= \sqrt{ 81 \kappa^2 \omega^2 (\alpha_1+2 \alpha_2 \omega)^2 - 12\kappa( \alpha_1  \omega+ \kappa)^3}.
\end{equation}  

The size structure at equilibrium is given by
\begin{equation}
  \begin{split}
  u(x) &= \frac{g(0)u(0)}{g(x)} \exp\left(- \int_0^x \frac{ z(\lambda,U) } {g(\lambda)}\,d\lambda\right)\\
  &= \frac{\alpha_1 \bar{V} + \alpha_2 \bar{W}}{\kappa(\omega-x)} \exp\left( -\int_0^x \frac{\beta + \gamma \bar{U}}{\kappa(\omega-\lambda)}\,d\lambda\right)\\
  &= \frac{\alpha_1 \bar{V} + \alpha_2 \bar{W}}{\kappa(\omega-x)} \exp\left( \left.\frac{(\beta + \gamma\bar{U})\log(\kappa(\omega-\lambda))}{\kappa}\right|_0^x\right)\\
  &= \frac{\alpha_1 \bar{V} + \alpha_2 \bar{W}}{\kappa(\omega-x)} \exp\left( \kappa^{-1}(\beta + \gamma\bar{U})\log(\kappa(\omega-x)) - \kappa^{-1}(\beta + \gamma\bar{U})\log(\kappa\omega)\right)\\
  &= \frac{\alpha_1 \bar{V} + \alpha_2 \bar{W}}{\kappa(\omega-x)} \frac{(\kappa(\omega-x))^{\frac{\beta+\gamma\bar{U}}{\kappa}}}{(\kappa\omega)^{\frac{\beta+\gamma\bar{U}}{\kappa}}}\\  
  &= (\alpha_1 \bar{V} + \alpha_2 \bar{W}) \frac{(\kappa(\omega-x))^{\frac{\beta+\gamma\bar{U}}{\kappa}-1}}{(\kappa\omega)^{\frac{\beta+\gamma\bar{U}}{\kappa}}}\\  
  &= (\alpha_1 \bar{V} + \alpha_2 \bar{W}) \frac{\kappa^{\frac{\beta+\gamma\bar{U}}{\kappa}-1}(\omega-x)^{\frac{\beta+\gamma\bar{U}}{\kappa}-1}}{\kappa^{\frac{\beta+\gamma\bar{U}}{\kappa}}\omega^{\frac{\beta+\gamma\bar{U}}{\kappa}}}\\  
  &= \frac{\left(\alpha_1 \bar{V} + \alpha_2 \bar{W}\right) (\omega-x)^{(\beta+\gamma\bar{U})/\kappa-1}}{\kappa \omega^{(\beta+\gamma\bar{U})/\kappa}}
  \end{split} 
\end{equation}
We also need initial conditions for our sensitivity PDEs. These are, by definition, $d u(x)/d\theta$. The derivations for these are contained in Appendix \ref{app:2}.

\subsection{Initial parameter values}
Consider now the initial values of the parameters for the optimisation algorithm. If we linearise the model by removing the density dependent natural mortality ($\gamma$) and consider the population at fished equilibrium, we have the following relation
\begin{equation}
  1=\int_0^\omega \frac{ \alpha_1 x + \alpha_2 x^2}{\kappa (\omega - x)} \exp\left(-\int_0^x \frac{\tilde{\beta} + f s(\lambda)}{\kappa(\omega-\lambda)}\,d\lambda\right)\,dx
\end{equation}
where $f$ is the constant fishing mortality and $\tilde{\beta}$ is a natural mortality term that is a mixture of density dependent and density independent effects. 



%But we also have
%\begin{equation}
%  1=\int_0^\omega \frac{ \alpha_1 x + \alpha_2 x^2}{\kappa (\omega - x)} \exp\left(-\int_0^x \frac{\beta + 0.4\gamma \bar{U}  + f s(\lambda)}{\kappa(\omega-\lambda)}\,d\lambda\right)\,dx
%\end{equation}

\begin{equation}
{{2\,\left(\int_{0}^{w}{{\frac{x^2\,e^ {- {{\int_{0}^{x}{{\frac{f s(\lambda)+\beta}{\kappa(\omega-\lambda)}}\;d\lambda}}} }}{\kappa(\omega-x)}}\;dx}
 \right)\,\left({{\int_{0}^{w}{{\frac{\left(\alpha_2 x^2+\alpha_1 x
 \right)\,e^ {- {\int_{0}^{x}{{\frac{f\,s\left(l\right)+\beta}{\kappa(\omega-
 \lambda)}}\;d\lambda}} }}{\kappa(\omega-x)}}\;dx}}}-1\right)}}
\end{equation}

At this equilibrium, we have the following size-structure
\begin{equation}
  \frac{\left(\omega-x\right)^{-1}\exp\left(-\int_0^x \frac{\tilde{\beta} + f s(\lambda)}{\kappa(\omega-\lambda)}\,d\lambda\right)}{
\int_0^\omega \left(\omega-x\right)^{-1} \exp\left(-\int_0^x \frac{\tilde{\beta} + f s(\lambda)}{\kappa(\omega-\lambda)}\,d\lambda\right)\,dx}        
\end{equation}
and thus the following catch structure
\begin{equation}
  h(x) = \frac{s(x)\left(\omega-x\right)^{-1}\exp\left(-\int_0^x \frac{\tilde{\beta} + f s(\lambda)}{\kappa(\omega-\lambda)}\,d\lambda\right)}{
\int_0^\omega s(x)\left(\omega-x\right)^{-1} \exp\left(-\int_0^x \frac{\tilde{\beta} + f s(\lambda)}{\kappa(\omega-\lambda)}\,d\lambda\right)\,dx}        
\end{equation}
and objective function
\begin{equation}
  G_1(\theta) = \int_0^\omega l(x) \log\left(\frac{l(x)}{h(x)}\right) \,dx.
\end{equation}
where $l(x)$ is the observed size-structure for the entire time period. We will also use the derivatives:
\begin{equation}
  \frac{d G_1}{d \beta} = \int_0^\omega -\frac{l(x)}{h(x)} \frac{ A(x) \left( - \int_0^x \frac{d\lambda}{\kappa(\omega-\lambda)}\right) B - A(x) \int_0^\omega A(x) \left( - \int_0^x \frac{d\lambda}{\kappa(\omega-\lambda)}\right)\,dx }{B^2}\,dx  
\end{equation}
where
\begin{equation}
  A(x) = s(x)(\omega-x)^{-1} \exp\left(-\int_0^x \frac{\tilde{\beta} + f s(\lambda)}{\kappa(\omega-\lambda)}\,d\lambda\right)
\end{equation}
and 
\begin{equation}
  B = \int_0^\omega s(x)\left(\omega-x\right)^{-1} \exp\left(-\int_0^x \frac{\tilde{\beta} + f s(\lambda)}{\kappa(\omega-\lambda)}\,d\lambda\right)\,dx
\end{equation}
and 
\begin{equation}
  \frac{d G_1}{d f} = \int_0^\omega -\frac{l(x)}{h(x)} \frac{ A(x) \left( - \int_0^x \frac{s(\lambda)d\lambda}{\kappa(\omega-\lambda)}\right) B - A(x) \int_0^\omega A(x) \left( - \int_0^x \frac{s(\lambda)d\lambda}{\kappa(\omega-\lambda)}\right)\,dx }{B^2}  \,dx
\end{equation}
We can simplify this further to 
\begin{equation}
  \frac{d G_1}{d \beta} = \int_0^\omega -l(x) \frac{ \left( - \int_0^x \frac{d\lambda}{\kappa(\omega-\lambda)}\right) B - \int_0^\omega A(x) \left( - \int_0^x \frac{d\lambda}{\kappa(\omega-\lambda)}\right)\,dx }{B}\,dx  
\end{equation}
and
\begin{equation}
  \frac{d G_1}{d f} = \int_0^\omega -l(x) \frac{ \left( - \int_0^x \frac{s(\lambda)d\lambda}{\kappa(\omega-\lambda)}\right) B - \int_0^\omega A(x) \left( - \int_0^x \frac{s(\lambda)d\lambda}{\kappa(\omega-\lambda)}\right)\,dx }{B}  \,dx
\end{equation}
\subsection{Optimisation algorithm}
The optimisation algorithm is essentially a BFGS (Broyden-Fletcher-Goldfarb-Shanno) method, but treated as a bi-level optimisation where $\iota$ is solved in the inner loop. This is because $\iota$ is a `nuisance parameter', in the following senses: \emph{i}) it is not of great interest directly, \emph{ii}) it is found through a different objective function to the other parameters, and \emph{iii}) that objective function could in principle be minimised to arbitrary precision (e.g. by taking $\iota$ to be a vector of spline coefficients) without over-parameterising the objective functions of interest. Therefore we let $\tilde{\theta}$ be the \emph{non-}$\iota$ components of $\theta$, so that $\theta = \langle\tilde{\theta},\iota\rangle$. Also let the main objective function be a weighted sum of the three `interesting' objective functions: $H_*(\theta) = \xi_2 H_2(\theta) +\xi_3 H_3(\theta) + \xi_4 H_4(\theta)$. We discuss how the $\xi_i$ are chosen later.

Then we have the derivative of the main objective function, 
\begin{equation}
  \nabla H_*(\tilde{\theta};\iota) \equiv \left\langle \xi_2 \frac{d H_2}{d \alpha_1}, \xi_2 \frac{d H_2}{d \alpha_2}, \xi_2 \frac{d H_2}{d \beta}, \xi_2 \frac{d H_2}{d \gamma}, \xi_2 \frac{d H_2}{d \kappa} + \xi_3 \frac{d H_3}{d \kappa} + \xi_4 \frac{d H_4}{d \kappa}, \xi_2 \frac{d H_2}{d \omega} + \xi_3 \frac{d H_3}{d \omega} + \xi_4 \frac{d H_4}{d \omega}\right\rangle
\end{equation}

\begin{algorithm}
  \caption{Optimisation algorithm}\label{alg:opt}
  \begin{algorithmic}
    \State $B_0 \gets I$, $\tilde{\theta}_0 \gets \tilde{\theta}_\text{ini}$, $k \gets 0$
    \State $\iota \gets \argmin_\iota H_1(\iota;\tilde{\theta}_{0})$
    \While{$|| \nabla H_*(\tilde{\theta}_k;\iota) || < \epsilon$}
    \State $\mathbf{q}_k \gets - B_k \nabla H_*(\tilde{\theta}_k;\iota)$
    \State Find $\zeta_k$ that satisfies \ref{eq:w1} and \ref{eq:w2}
    \State $\mathbf{s}_k \gets \zeta_k \mathbf{q}_k$
    \State $\tilde{\theta}_{k+1} \gets \tilde{\theta}_k + \mathbf{s}_k$
    \State $\mathbf{y}_k \gets \nabla H_*(\tilde{\theta}_{k+1};\iota) - \nabla H_*(\tilde{\theta}_k;\iota)$
    \State $\rho_k \gets (\mathbf{s_k}' \mathbf{y}_k)^{-1}$
    \State $B_{k+1} \gets (I - \rho_k \mathbf{s}_k \mathbf{y}_k') B_k (I - \rho_k \mathbf{y}_k' \mathbf{s}_k) + \rho_k \mathbf{s}_k\mathbf{s}_k'$
    \State $k \gets k+1$
    \State $\iota \gets \argmin_\iota H_1(\iota;\tilde{\theta}_{k})$
    \EndWhile
%    \If{k=1}
%    \State $\zeta_k \gets 1/ || \nabla H(\theta_k)
  \end{algorithmic}
\end{algorithm}

Wolf conditions:
\begin{subequations}
  \begin{align}
    H(\theta_{k+1}) &\leq H(\theta_k) + c_1 \zeta_k \nabla H(\theta_k)' \mathbf{q}_k \label{eq:w1}\\
    \nabla H(\theta_{k+1})' &\mathbf{q}_k \geq c_2 \nabla H(\theta_k)' \mathbf{q}_k\label{eq:w2}
  \end{align}
\end{subequations}

\section{Application}
\begin{table}
  \centering
\begin{tabular}{|c|c|c|c|}
\hline
Process & Symbol & Units & Value\\
\hline
\multirow{2}{*}{Birth}
 & $\alpha_1$  & $\text{cm}^{-1}\text{yr}^{-1}$  & \\
 & $\alpha_2$  & $\text{cm}^{-1}\text{yr}^{-1}$  & \\
\hline
\multirow{2}{*}{Growth}
 & $\kappa$   & $\text{yr}^{-1}$  & \\
 & $\omega$ & $\text{cm}$ & \\
\hline
\multirow{2}{*}{Death}
 & $\beta$  & $\text{yr}^{-1}$ & \\
 & $\gamma$  & $\text{fish}^{-1}\text{yr}^{-1}$  & \\
\hline 
\end{tabular}
\caption{Active parameters}
\label{tbl:active}
\end{table}

\clearpage
\begin{appendices}  
\section{Analytical Equilibrium Derivations}\label{app:eq}
\emph{Part one - Sub \ref{seq:2a} and \ref{seq:2b} into \ref{eq:1.1} and integrate over $x$ from 0 to $\omega$.}

\begin{subequations}
  \begin{align}
    \int_0^{\omega} u_t(x,t)\,dx + \int_0^{\omega} [\kappa(\omega-x)u(x,t)]_x \, dx &= \int_0^{\omega} -(\beta+\gamma U(t)) u(x,t)\, dx\\
    \Rightarrow \dot{U}(t) + \int_0^{\omega} \kappa(\omega-x)u_x(x,t)\,dx - \int_0^{\omega} \kappa u(x,t)\,dx&= -(\beta+\gamma U(t)) U(t)\\
    \Rightarrow \dot{U}(t) + \int_0^{\omega} \kappa(\omega-x)u_x(x,t)\,dx - \kappa  U(t) &= -(\beta+\gamma U(t)) U(t)\label{eq:1pt3}
  \end{align}
\end{subequations}

Integrating the second term in \ref{eq:1pt3} by parts:
\begin{equation*}
  \upsilon=\kappa(\omega - x),~~~d\nu=u_x(x,t)\,dx,~~~d\upsilon=-\kappa\,dx,~~~\nu=u(x,t)
\end{equation*}

\begin{subequations}
  \begin{align*}
    &\kappa(\omega-x) u(x,t)]_0^{\omega} - \int_0^{\omega} u(x,t) (-\kappa)\,dx\\
    &\Rightarrow \kappa(\omega - \omega)u(\omega,t) - \kappa(\omega-0)u(0,t) + \kappa U(t)\\
      &\Rightarrow \kappa U(t) - \kappa(\omega-0)u(0,t)
  \end{align*}
\end{subequations}
subbing back:
\begin{subequations}
  \begin{align}
    &\dot{U}(t) + \kappa U(t) - \kappa(\omega-0)u(0,t) - \kappa  U(t) = -(\beta+\gamma U(t)) U(t)\\
    &\Rightarrow \dot{U}(t) = -(\beta+\gamma U(t)) U(t) + \kappa(\omega-0)u(0,t)\\
    &\Rightarrow \dot{U}(t) = -(\beta+\gamma U(t)) U(t) + \int_0^\omega b(x) u(x,t)\, dx\\
    &\Rightarrow \dot{U}(t) = -(\beta+\gamma U(t)) U(t) + \int_0^\omega \left( \alpha_1 x u(x,t) + \alpha_2 x^2 u(x,t) \right) \, dx\\
    &\Rightarrow \dot{U}(t) = -(\beta+\gamma U(t)) U(t) + \int_0^\omega \alpha_1 x u(x,t)\, dx + \int_0^\omega \alpha_2 x^2 u(x,t) \, dx\\
    &\Rightarrow \dot{U}(t) = -(\beta+\gamma U(t)) U(t) + \alpha_1 V(t) + \alpha_2 W(t)
  \end{align}
\end{subequations}

\clearpage
\emph{Part two - sub \ref{seq:2a} and \ref{seq:2b} into \ref{eq:1.1}, multiply by $x$ and integrate over $x$ from 0 to $\omega$.}

\begin{subequations}
  \begin{align}
    \int_0^{\omega} u_t(x,t)x\,dx + \int_0^{\omega} [\kappa(\omega-x)u(x,t)]_x x \, dx &= \int_0^{\omega} -(\beta+\gamma U(t)) u(x,t)x\, dx\\
    \Rightarrow \dot{V}(t) + \int_0^{\omega} x\kappa(\omega-x)u_x(x,t)\,dx - \int_0^{\omega} x\kappa u(x,t)\,dx &= -(\beta+\gamma U(t)) V(t)\\
    \Rightarrow \dot{V}(t) + \int_0^{\omega} x\kappa(\omega-x)u_x(x,t)\,dx - \kappa V(t) &= -(\beta+\gamma U(t)) V(t)\label{eq:2pt3}
  \end{align}
\end{subequations}
Integrating the second term in \ref{eq:2pt3} by parts:
\begin{equation*}
  \upsilon=x\kappa(\omega - x),~~~d\nu=u_x(x,t)\,dx,~~~d\upsilon=\kappa \omega - 2\kappa x \,dx,~~~\nu=u(x,t)
\end{equation*}

\begin{subequations}
  \begin{align*}
    &\left. x\kappa(\omega - x) u(x,t) \right]_0^{\omega} - \int_0^{\omega} u(x,t) \left[ \kappa \omega - 2 \kappa x \right] \,dx\\
    &\Rightarrow \left. x\kappa(\omega -x) u(x,t) \right]_0^{\omega} - \int_0^{\omega} \kappa \omega u(x,t)\,dx + \int_0^{\omega} 2\kappa x u(x,t)\,dx\\
    &\Rightarrow 2\kappa V(t) - \kappa \omega U(t) 
  \end{align*}
\end{subequations}

subbing back into \ref{eq:2pt3}
\begin{subequations}
  \begin{align*}
    &\dot{V}(t) + 2\kappa V(t) - \kappa \omega U(t) - \kappa V(t) = -(\beta+\gamma U(t)) V(t)\\
    &\Rightarrow \dot{V}(t) + \kappa V(t) - \kappa \omega U(t) = -(\beta+\gamma U(t)) V(t)\\
    &\Rightarrow \dot{V}(t) = -(\beta+\gamma U(t)) V(t) + \kappa \omega U(t) - \kappa V(t)
  \end{align*}
\end{subequations}

\clearpage
\emph{Part three - sub \ref{seq:2a} and \ref{seq:2b} into \ref{eq:1.1}, multiply by $x^2$ and integrate over $x$ from 0 to $\omega$.}

\begin{subequations}
  \begin{align}
    \int_0^{\omega} u_t(x,t)x^2\,dx + \int_0^{\omega} [\kappa(\omega-x)u(x,t)]_x x^2 \, dx &= \int_0^{\omega} -(\beta+\gamma U(t)) u(x,t)x^2\, dx\\
    \Rightarrow \dot{W}(t) + \int_0^{\omega} x^2\kappa(\omega-x)u_x(x,t)\,dx - \int_0^{\omega} x^2\kappa u(x,t)\,dx &= -(\beta+\gamma U(t)) W(t)\\
    \Rightarrow \dot{W}(t) + \int_0^{\omega} x^2\kappa(\omega-x)u_x(x,t)\,dx - \kappa W(t) &= -(\beta+\gamma U(t)) W(t)\label{eq:2pt3}
  \end{align}
\end{subequations}
Integrating the second term in \ref{eq:2pt3} by parts:
\begin{equation*}
  \upsilon=x^2\kappa(\omega - x),~~~d\nu=u_x(x,t)\,dx,~~~d\upsilon=2\kappa \omega x - 3\kappa x^2 \,dx,~~~\nu=u(x,t)
\end{equation*}

\begin{subequations}
  \begin{align*}
    &\left. x^2 \kappa(\omega - x) u(x,t) \right]_0^{\omega} - \int_0^{\omega} u(x,t) \left[ 2\kappa \omega x - 3 \kappa x^2 \right] \,dx\\
    &\Rightarrow \left. x\kappa(\omega -x) u(x,t) \right]_0^{\omega} - \int_0^{\omega} 2\kappa \omega x u(x,t)\,dx + \int_0^{\omega} 3\kappa x^2 u(x,t)\,dx\\
    &\Rightarrow 3\kappa W(t) - 2 \kappa \omega V(t) 
  \end{align*}
\end{subequations}

subbing back into \ref{eq:2pt3}
\begin{subequations}
  \begin{align*}
    &\dot{W}(t) + 3\kappa W(t) - 2\kappa \omega V(t) - \kappa W(t) = -(\beta+\gamma U(t)) W(t)\\
    &\Rightarrow \dot{W}(t) + 2\kappa W(t) - 2\kappa \omega V(t) = -(\beta+\gamma U(t)) W(t)\\
    &\Rightarrow \dot{W}(t) = -(\beta+\gamma U(t)) W(t) + 2\kappa \omega V(t) - 2\kappa W(t)
  \end{align*}
\end{subequations}

\clearpage
\section{Initial state for sensitivity PDEs}\label{app:2}


%%%%%%%%%%%%%% a_1

We can simplify function $U(x)$ in a following way
\begin{equation}\label{eq:U}
 \begin{split}
  U(x) &=  \frac{ \left(\alpha_1\bar{V}+\alpha_2\bar{W}\right)(\omega-x)^{(\beta+\gamma\bar{U})/\kappa -1} }{\kappa\omega^{(\beta+\gamma\bar{U})/\kappa}} \\ 
  &=\frac{\alpha_1\bar{V}+\alpha_2\bar{W}}{\kappa(\omega-x)}\left(1-\frac{x}{\omega}\right)^{(\beta+\gamma\bar{U})/\kappa}\\  
  &=\frac{\alpha_1 \frac{\kappa\omega (Z-\beta-\kappa)}{\gamma Z} +\alpha_2\frac{2\kappa^2\omega^2(Z-\beta-\kappa)}{\gamma Z (Z+k)}}{\kappa(\omega-x)}\left(1-\frac{x}{\omega}\right)^{(Z-\kappa)/\kappa}\\
    &=\frac{Z-\beta-\kappa}{\gamma Z}\left(\alpha_1 + \frac{2\alpha_2 \kappa\omega}{Z+k}\right)\left(1-\frac{x}{\omega}\right)^{Z/\kappa-2}.
  \end{split}
\end{equation}

{\bf 1.} Now for the first parameter we have

\begin{equation}\label{eq:da1}
\begin{split}
  \frac{d u(x)}{d\alpha_1} =& \frac{(\beta+\kappa)Z_{\alpha_1}}{\gamma Z^2}\left(\alpha_1 + \frac{2\alpha_2 \kappa\omega}{Z+\kappa}\right)\left(1-\frac{x}{\omega}\right)^{Z/\kappa-2}+\\
&  \frac{Z-\beta-\kappa}{\gamma Z}\left(1-\frac{2\alpha_2\kappa\omega}{(Z+\kappa)^2}Z_{\alpha_1}\right)\left(1-\frac{x}{\omega}\right)^{Z/\kappa-2}+\\ 
& \frac{Z-\beta-\kappa}{\gamma Z}\left(\alpha_1+ \frac{2\alpha_2 \kappa\omega}{Z+\kappa}
\right)\left(1-\frac{x}{\omega}\right)^{Z/\kappa-2}\log\left(1-\frac{x}{\omega}\right)\frac{Z_{\alpha_1}}{\kappa}, 
  \end{split}
\end{equation}
which can be written as
\begin{equation}\begin{split}
\left(1-\frac{x}{\omega}\right)^{\frac{Z}{\kappa}-2}&\left[ 
\frac{Z-\beta-\kappa}{\gamma Z}\left(1-\frac{2\alpha_2\kappa\omega}{(Z+\kappa)^2}Z_{\alpha_1}\right) + \right.\\
&\left.+\frac{Z_{\alpha_1}}{\gamma Z}\left(\alpha_1 + \frac{2\alpha_2 \kappa\omega}{Z+\kappa}\right)
  \left(\frac{\beta+\kappa}{Z}+ \log\left(1-\frac{x}{\omega}\right)\frac{Z-\beta-\kappa}{\kappa}\right) \right], 
\end{split}\end{equation}
where
\begin{equation}\begin{split}
Z_{\alpha_1}= &\frac{\kappa^2 \omega\,(3\zeta -6(\kappa+\alpha_1 \omega)^2 + 27\kappa\omega( \alpha_1  + 2 \alpha_2 \omega))}
{\zeta\, (9 \alpha_1 \kappa^2 \omega + 18 \alpha_2 \kappa^2 \omega^2 + \kappa \zeta)^{2/3}}\left(
\frac1{2^{1/3}3^{2/3}}-\right.\\
&\left. \frac{(2/3)^{1/3} \kappa ( \kappa + \alpha_1\omega)}{(9 \alpha_1 \kappa^2 \omega + 18 \alpha_2 \kappa^2 \omega^2 + \kappa \zeta)^{2/3}}\right)+\frac{(2/3)^{1/3}\kappa\omega}{(9 \alpha_1 \kappa^2 \omega + 18 \alpha_2 \kappa^2 \omega^2 + \kappa \zeta)^{1/3}}.
\end{split}\end{equation}


%%%%%%%%%%%%%%%%%%% a_2


{\bf 2.} Similarly

\begin{equation}\label{eq:da2}
\begin{split}
  \frac{d u(x)}{d\alpha_2} =& \frac{(\beta+\kappa)Z_{\alpha_2}}{\gamma Z^2}\left(\alpha_1 + \frac{2\alpha_2 \kappa\omega}{Z+\kappa}\right)\left(1-\frac{x}{\omega}\right)^{Z/\kappa-2}+\\
&  \frac{Z-\beta-\kappa}{\gamma Z} \left(\frac{2\kappa\omega}{Z+\kappa}-\frac{2\alpha_2\kappa\omega}{(Z+\kappa)^2}Z_{\alpha_2}\right)\left(1-\frac{x}{\omega}\right)^{Z/\kappa-2}+\\ 
& \frac{Z-\beta-\kappa}{\gamma Z}\left(\alpha_1+ \frac{2\alpha_2 \kappa\omega}{Z+\kappa}
\right)\left(1-\frac{x}{\omega}\right)^{Z/\kappa-2}\log\left(1-\frac{x}{\omega}\right)\frac{Z_{\alpha_2}}{\kappa}, 
  \end{split}
\end{equation}
that is simplified
\begin{equation}\begin{split}
\left(1-\frac{x}{\omega}\right)^{\frac{Z}{\kappa}-2}&\left[ 
 \frac{Z-\beta-\kappa}{\gamma Z} \left(\frac{2\kappa\omega}{Z+\kappa}-\frac{2\alpha_2\kappa\omega}{(Z+\kappa)^2}Z_{\alpha_2}\right) +\right.\\
&\left. +\frac{Z_{\alpha_2}}{\gamma Z}\left(\alpha_1 + \frac{2\alpha_2 \kappa\omega}{Z+\kappa}\right) \left(\frac{\beta+\kappa}{Z}+ \log\left(1-\frac{x}{\omega}\right)\frac{Z-\beta-\kappa}{\kappa}\right) \right], 
\end{split}\end{equation}

where
\begin{equation}
Z_{\alpha_2}= \frac{6 \kappa^2 \omega^2(\zeta + 9 \kappa \omega (\alpha_1 +2 \alpha_2 \omega))}{\zeta\,(9 \alpha_1 \kappa^2 \omega + 18 \alpha_2 \kappa^2 \omega^2 + \kappa \zeta)^{2/3}} \left( \frac1{2^{1/3}3^{2/3}}-\frac{  (2/3)^{1/3} \kappa( \kappa +  \alpha_1  \omega)}{(9 \alpha_1 \kappa^2 \omega + 18 \alpha_2 \kappa^2 \omega^2 + \kappa \zeta)^{2/3}} \right).
\end{equation}

%%%%%%%%%%%%%%%%%%%%%%%%%%%%%%%%%%% k

\bigskip

{\bf 3.} The next one is
\begin{equation}\label{eq:dk}
\begin{split}
  \frac{d u(x)}{d\kappa} =& \frac{(\beta+\kappa)Z_{\kappa}-Z}{\gamma Z^2}\left(\alpha_1 + \frac{2\alpha_2 \kappa\omega}{Z+\kappa}\right)\left(1-\frac{x}{\omega}\right)^{Z/\kappa-2}+\\
&  \frac{Z-\beta-\kappa}{\gamma Z} \left(\frac{2\alpha_2\omega}{Z+\kappa}-\frac{2\alpha_2\kappa\omega}{(Z+\kappa)^2}(Z_{\kappa}+1)\right)\left(1-\frac{x}{\omega}\right)^{Z/\kappa-2}+\\ 
& \frac{Z-\beta-\kappa}{\gamma Z}\left(\alpha_1+ \frac{2\alpha_2 \kappa\omega}{Z+\kappa}
\right)\left(1-\frac{x}{\omega}\right)^{Z/\kappa-2}\log\left(1-\frac{x}{\omega}\right)\left(\frac{Z_{\kappa}}{\kappa}-\frac{Z}{\kappa^2}\right), 
  \end{split}
\end{equation}
that is 
\begin{equation}\begin{split}
&\left(1-\frac{x}{\omega}\right)^{\frac{Z}{\kappa}-2} \left[ 
\frac{Z_{\kappa}}{\gamma Z }\left(\left(\alpha_1 + \frac{2\alpha_2 \kappa\omega}{Z+\kappa}\right)
  \left(\frac{\beta+\kappa}{Z}+ \log\left(1-\frac{x}{\omega}\right)\frac{Z-\beta-\kappa}{\kappa}\right)-\frac{2\alpha_2\kappa\omega(Z-\beta-\kappa)}{(Z+\kappa)^2} \right)+ \right.\\
  &\left.  \frac{Z-\beta-\kappa}{\gamma Z (Z+\kappa)}\left(2\alpha_2\omega + \frac{2\alpha_2\kappa\omega}{Z+\kappa}\right) - \left(\alpha_1 + \frac{2\alpha_2 \kappa\omega}{Z+\kappa}\right)\left(\frac1{\gamma Z}+\log\left(1-\frac{x}{\omega}\right)\frac{Z-\beta-\kappa}{\gamma\kappa^2} \right)       \right]
  \end{split}
\end{equation}

where
\begin{equation}\begin{split}
Z_{\kappa}=& \frac{6 \kappa\, (\alpha_1  \omega \zeta + 2 \alpha_2  \omega^2 \zeta - 
      (2 \kappa + \alpha_1 \omega) ( \kappa + \alpha_1  \omega)^2 + 9 \kappa \omega^2 ( \alpha_1 + 2 \alpha_2  \omega)^2)}{\zeta\, (9 \alpha_1 \kappa^2 \omega + 18 \alpha_2 \kappa^2 \omega^2 +\kappa \zeta)^{2/3}}\left( \frac{1}{ 2^{1/3}3^{2/3}} -\right. \\
    &\left.\frac{(2/3)^{1/3} \kappa ( \kappa +  \alpha_1  \omega)}{(9 \alpha_1 \kappa^2 \omega + 18 \alpha_2 \kappa^2 \omega^2 +\kappa \zeta)^{2/3}} \right) + \frac{ (2/3)^{1/3}( 2 \kappa +  \alpha_1 \omega)}{ (9 \alpha_1 \kappa^2 \omega + 18 \alpha_2 \kappa^2 \omega^2 +\kappa \zeta)^{1/3}}.
  \end{split}\end{equation}

%%%%%%%%%%%%%%%%%%%%%%%%%%%%%%%%%%%%%%% omega

\bigskip

{\bf 4.} For the parameter $\omega$ we have
\begin{equation}\label{eq:dw}
\begin{split}
  \frac{d u(x)}{d\omega} =& \frac{(\beta+\kappa)Z_{\omega}}{\gamma Z^2}\left(\alpha_1 + \frac{2\alpha_2 \kappa\omega}{Z+\kappa}\right)\left(1-\frac{x}{\omega}\right)^{Z/\kappa-2}+\\
&  \frac{Z-\beta-\kappa}{\gamma Z} \left(\frac{2\alpha_2\kappa}{Z+\kappa}-\frac{2\alpha_2\kappa\omega}{(Z+\kappa)^2}Z_{\omega}\right)\left(1-\frac{x}{\omega}\right)^{Z/\kappa-2}+\\ 
& \frac{Z-\beta-\kappa}{\gamma Z}\left(\alpha_1+ \frac{2\alpha_2 \kappa\omega}{Z+\kappa}
\right)\left(1-\frac{x}{\omega}\right)^{Z/\kappa-2}\left(\frac{Z_{\omega}}{\kappa}\log\left(1-\frac{x}{\omega}\right)+\frac{x(Z-2\kappa)}{\kappa\omega(\omega- x)}\right), 
  \end{split}
\end{equation}
which is 
\begin{equation}\begin{split}
\left(1-\frac{x}{\omega}\right)^{\frac{Z}{\kappa}-2}&\left[ 
 \frac{Z-\beta-\kappa}{\gamma Z} \left(\frac{2\kappa\omega}{Z+\kappa}-\frac{2\alpha_2\kappa\omega}{(Z+\kappa)^2}Z_{\omega} + \left(\alpha_1+ \frac{2\alpha_2 \kappa\omega}{Z+\kappa}
\right) \frac{x(Z-2\kappa)}{\kappa\omega(\omega- x)} \right)+ \right.\\
&\left. +\frac{Z_{\omega}}{\gamma Z}\left(\alpha_1 + \frac{2\alpha_2 \kappa\omega}{Z+\kappa}\right) \left(\frac{\beta+\kappa}{Z}+\log\left(1-\frac{x}{\omega}\right)\frac{Z-\beta-\kappa}{\kappa}\right)  \right], 
\end{split}\end{equation}
where
\begin{equation}\begin{split}
Z_{\omega}=& \frac{3 \kappa^2 ( \alpha_1 \zeta + 4 \alpha_2  \omega \zeta - 2 \alpha_1  (\kappa + \alpha_1 \omega)^2 + 
   18 \alpha_2 \kappa \omega^2 (\alpha_1 + 2 \alpha_2 \omega) + 9 \kappa \omega (\alpha_1 + 2 \alpha_2 \omega)^2)}{\zeta\, (9 \alpha_1 \kappa^2 \omega + 18 \alpha_2 \kappa^2 \omega^2 +\kappa \zeta)^{2/3}}\left( \frac{1}{ 2^{1/3}3^{2/3}} -\right. \\
    &\left.\frac{(2/3)^{1/3} \kappa ( \kappa +  \alpha_1  \omega)}{(9 \alpha_1 \kappa^2 \omega + 18 \alpha_2 \kappa^2 \omega^2 +\kappa \zeta)^{2/3}} \right) + \frac{ (2/3)^{1/3} \alpha_1 \kappa}{ (9 \alpha_1 \kappa^2 \omega + 18 \alpha_2 \kappa^2 \omega^2 +\kappa \zeta)^{1/3}}.
  \end{split}\end{equation}


%%%%%%%%%%%%%%%%%%%%%%%%%%%%%%%%%%%%%%% beta

\bigskip

{\bf 5.} Next parameter is $\beta$ which is much simpler since $Z$ isn't function of $\beta$ so derivative is just
\begin{equation}\label{eq:db}
  \frac{d u(x)}{d\beta} =-\frac{1}{\gamma Z}\left(\alpha_1 + \frac{2\alpha_2 \kappa\omega}{Z+k}\right)\left(1-\frac{x}{\omega}\right)^{Z/\kappa-2}.
\end{equation}

%%%%%%%%%%%%%%%%%%%%%%%%%%%%%%%%%%%%%%% gamma

\bigskip

{\bf 6.} Very similar is for the $\gamma$
\begin{equation}\label{eq:dg}
  \frac{d u(x)}{d\gamma} =-\frac{Z-\beta-\kappa}{\gamma^2 Z}\left(\alpha_1 + \frac{2\alpha_2 \kappa\omega}{Z+k}\right)\left(1-\frac{x}{\omega}\right)^{Z/\kappa-2}.
\end{equation}

\end{appendices}
\bibliography{spade}
\end{document}
